\documentclass[fleqn, 11pt]{article}
\usepackage[spanish]{babel}
\usepackage[utf8]{inputenc}
\usepackage{amsmath, amssymb, amsfonts}
\usepackage{parskip}
\usepackage{lipsum}
\usepackage[left = 18mm, right = 18mm, bottom = 18mm, top = 18mm]{geometry}
\usepackage{tcolorbox}
\tcbuselibrary{theorems}
\tcbuselibrary{breakable}
\usepackage{colortbl}
\usepackage{array, tabularx}
\usepackage[pdftex, hidelinks]{hyperref}
\usepackage{multirow}
\usepackage{multicol}
\usepackage{enumerate}
\usepackage{graphicx}

\usepackage[proportional,scaled=1.064]{erewhon}
\usepackage[erewhon,vvarbb,bigdelims]{newtxmath}
\usepackage[T1]{fontenc}
\renewcommand*\oldstylenums[1]{\textosf{#1}}

\tcbsetforeverylayer{autoparskip}

\tcbset{theorem full label supplement={hypertarget={#1}}}

\newtcbtheorem[auto counter, number freestyle = {\noexpand \arabic{\tcbcounter}.}]{definicion}{D\hspace{0.1mm}e\hspace{0.1mm}f\hspace{0.4mm}i\hspace{0.1mm}n\hspace{0.1mm}i\hspace{0.1mm}c\hspace{0.1mm}i\hspace{0.1mm}ó\hspace{0.1mm}n}{colback = white, colframe = black, fonttitle = \bfseries, separator sign = {\hspace{2mm}}}{def}

\newtcbtheorem[auto counter, number freestyle = {\noexpand \arabic{\tcbcounter}.}]{teorema}{T\hspace{0.1mm}e\hspace{0.1mm}o\hspace{0.1mm}r\hspace{0.1mm}e\hspace{0.1mm}m\hspace{0.1mm}a}{colback = white, colframe = black, fonttitle = \bfseries, separator sign = {\hspace{2mm}}}{teo}

\newtcbtheorem[use counter from = teorema]{proposicion}{P\hspace{0.1mm}r\hspace{0.1mm}o\hspace{0.1mm}p\hspace{0.1mm}o\hspace{0.1mm}s\hspace{0.1mm}i\hspace{0.1mm}c\hspace{0.1mm}i\hspace{0.1mm}ó\hspace{0.1mm}n}{colback = white, colframe = black, fonttitle = \bfseries, separator sign = {\hspace{2mm}}}{prop}

\newtcbtheorem[use counter from = teorema]{corolario}{C\hspace{0.1mm}o\hspace{0.1mm}r\hspace{0.1mm}o\hspace{0.1mm}l\hspace{0.1mm}a\hspace{0.1mm}r\hspace{0.1mm}i\hspace{0.1mm}o}{colback = white, colframe = black, fonttitle = \bfseries, separator sign = {\hspace{2mm}}}{cor}

\newtcbtheorem[use counter from = teorema]{lema}{L\hspace{0.1mm}e\hspace{0.1mm}m\hspace{0.1mm}a}{colback = white, colframe = black, fonttitle = \bfseries, separator sign = {\hspace{2mm}}}{lem}

\newtcbtheorem[auto counter, number freestyle = {\noexpand \arabic{\tcbcounter}.}]{ejemplo}{E\hspace{0.1mm}j\hspace{0.1mm}e\hspace{0.1mm}m\hspace{0.1mm}p\hspace{0.1mm}l\hspace{0.1mm}o}{colback = white, colframe = black, fonttitle = \bfseries, separator sign = {\hspace{2mm}}}{ejem}

\begin{document}
    \begin{definicion}[beforeafter skip = 4mm]{}{grafica_de_grados}
        Sean $ G_1 $ y $ G_2 $ gráficas tales que $ \textnormal{V}(G_1) \cap \textnormal{V}(G_2) = \emptyset $. Se define a $ G_1 \ast G_2 = (\textnormal{V}, \textnormal{A}) $ como la \textbf{gráfica de grados} con \vspace{2mm}
        
        $ \textnormal{V}(G_1 \ast G_2) = \textnormal{V}(G_1) \cup \textnormal{V}(G_2) $ \quad y \vspace{-1mm}
        \begin{align*}
            \hspace{-9mm} \textnormal{A}(G_1 \ast G_2) =& \left\{ uv \mid u \in \textnormal{V}(G_1), \, v \in \textnormal{V}(G_2), \textnormal{ gr}(u) = \delta(G_1) + l \phantom{|_i^i} \textnormal{y} \phantom{|_i^i} \textnormal{gr}(v) = \Delta(G_2) - l, \right. \\
            & \left. \phantom{|_i^i} \textnormal{con } l = 0,1, \ldots, \lvert \textnormal{V}(G_2) \rvert \right\} \cup  \left\{ uv \mid u \in \textnormal{V}(G_2), \, v \in \textnormal{V}(G_1), \textnormal{ gr}(u) = \delta(G_2) + l \phantom{|_i^i} \textnormal{y } \right. \\
            & \left. \phantom{|_i^i} \textnormal{gr}(v) = \Delta(G_1) - l, \textnormal{ con } l = 0,1, \ldots, \lvert \textnormal{V}(G_1) \rvert \right\}
        \end{align*}
    \end{definicion}

    \begin{ejemplo}[breakable, pad at break = 4mm, beforeafter skip = 3mm]{}{}
        Sean $ G_1 $ y $ G_2 $ gráficas como se muestra abajo. Para construir la gráfica de grados de $ G_1 $ y $ G_2 $, se puede seguir el siguiente procedimiento: \vspace{3mm}

        \begin{center}
            \begin{minipage}[h]{0.3\linewidth}
                \includegraphics[width=0.9\linewidth]{Ejemplo_1/Ejemplo1_G1.jpg}
            \end{minipage} \hspace{0.1\linewidth}
            \begin{minipage}[h]{0.3\linewidth}
                \includegraphics[width=0.9\linewidth]{Ejemplo_1/Ejemplo1_G2.jpg}
            \end{minipage}
        \end{center} \vspace{3mm}

        \begin{enumerate}
            \item Obtener el grado máximo y mínimo de cada gráfica. En este caso, $ \delta(G_1) = 1, \delta(G_2) = 2 $, \mbox{$ \Delta(G_1) = 2 $} y $ \Delta(G_2) = 4 $. \vspace{3mm}
            \item Elaborar dos tablas donde la primer fila esté conformada por los grados anteriormente obtenidos como se muestra a continuación: \vspace{3mm}
            
            \begin{center}
                \begin{minipage}[h]{0.3\linewidth}
                    \begin{tcolorbox}[title empty, center, colframe = black!99!white, colback = white, sharp corners, hbox, left = -0.9mm, right = -0.9mm, top = -0.9mm, bottom = -0.9mm]
                        \begin{tabular}{c|c}
                            \rowcolor{gray!36!white} 
                            $ \delta(G_1) = 1 $ & $ \Delta(G_2) = 4 $ 
                        \end{tabular}
                    \end{tcolorbox}
                \end{minipage}
                \begin{minipage}[h]{0.3\linewidth}
                    \begin{tcolorbox}[title empty, center, colframe = black!99!white, colback = white, sharp corners, hbox, left = -0.9mm, right = -0.9mm, top = -0.9mm, bottom = -0.9mm]
                        \begin{tabular}{c|c}
                            \rowcolor{gray!36!white} 
                            $ \delta(G_2) = 2 $ & $ \Delta(G_1) = 2 $ 
                        \end{tabular}
                    \end{tcolorbox}
                \end{minipage}
            \end{center} \vspace{3mm}
            
            \item En las columnas de las tablas que corresponden a los grados máximos de cada gráfica, se coloca el número de la celda anterior disminuido en uno, hasta llegar al cero. Mientras que las columnas que corresponden a los grados mínimos se coloca el número de la celda anterior aumentado en uno. \vspace{3mm}
            
            \begin{center}
                \begin{minipage}[h]{0.3\linewidth}
                    \begin{tcolorbox}[title empty, center, colframe = black!99!white, colback = white, sharp corners, hbox, nobeforeafter, left = -0.9mm, right = -0.9mm, top = -0.9mm, bottom = -0.9mm]
                        \begin{tabular}{c|c}
                            \rowcolor{gray!36!white} 
                            $ \delta(G_1) = 1 $ & $ \Delta(G_2) = 4 $ \\ \hline\hline
                            $ 2 $               & $ 3 $ \\ \hline
                            $ 3 $               & $ 2 $ \\ \hline
                            $ 4 $               & $ 1 $ \\ \hline
                            $ 5 $               & $ 0 $
                        \end{tabular}
                    \end{tcolorbox}
                \end{minipage}
                \begin{minipage}[h]{0.3\linewidth}
                    \begin{tcolorbox}[title empty, center, colframe = black!99!white, colback = white, sharp corners, hbox, nobeforeafter, left = -0.9mm, right = -0.9mm, top = -0.9mm, bottom = -0.9mm]
                        \begin{tabular}{c|c}
                            \rowcolor{gray!36!white} 
                            $ \delta(G_2) = 2 $ & $ \Delta(G_1) = 2 $ \\ \hline\hline
                            $ 3 $               & $ 1 $ \\ \hline
                            $ 4 $               & $ 0 $
                        \end{tabular}
                    \end{tcolorbox}
                \end{minipage}
            \end{center} \vspace{3mm}

            \item Por último, se dibujan los vértices de ambas gráficas y se hacen adyacentes aquellos que tengan el grado indicado en cada fila, en su respectiva gráfica.
            
            \begin{center}
                \includegraphics[width=0.3\linewidth]{Ejemplo_1/Ejemplo1_Gg.jpg}
            \end{center}
        \end{enumerate}
    \end{ejemplo}

    \begin{proposicion}[beforeafter skip = 4mm]{}{Gg_bipartita}
        Si $ G_1 $ y $ G_2 $ son gráficas, entonces $ G_1 \ast G_2 $ es bipartita.
    \end{proposicion}

    \begin{proposicion}[beforeafter skip = 4mm]{}{conmutatividad}
            Si $ G_1 $ y $ G_2 $ son gráficas, entonces $ G_1 \ast G_2 = G_2 \ast G_1 $.
        \end{proposicion}

    \begin{teorema}[beforeafter skip = 4mm]{}{subgrafica_bipartita_completa}
        Sean $ G_1 $ y $ G_2 $ gráficas tal que $ G_1 $ es $ r $-regular. Si $ V' = \left\lbrace v \in \textnormal{V}(G_2) \mid \textnormal{gr}(v) = \delta(G_2) \textnormal{ gr}(v) = \Delta(G_2) \right\rbrace $, entonces \vspace{3mm}
        
        \begin{enumerate}[i)]
            \item la subgráfica inducida de $ G_1 \ast G_2 $ por $ \textnormal{V}(G_1) \cup V' $ es bipartita completa.
            \item Todo $ v \in \textnormal{V}(G_2) \setminus V' $ es aislado.
        \end{enumerate}
    \end{teorema}

    \begin{corolario}[beforeafter skip = 4mm]{}{maxmin_bipartita_completa}
        Sean $ G_1 $ y $ G_2 $ gráficas. Si $ G_1 $ es $ r $-regular y $ \forall v \in \textnormal{V}(G_2) $ se da que gr$(v) = \delta(G_2) $ ó gr$(v) = \Delta(G_2) $, entonces $ G_1 \ast G_2 $ es bipartita completa.
    \end{corolario}

    \begin{corolario}[beforeafter skip = 4mm]{}{regulares_completa}
        Si $ G_1 $ y $ G_2 $ son gráficas $ r $-regular y $ s $-regular, respectivamente, entonces $ G_1 \ast G_2 $ es bipartita completa.
    \end{corolario}

    \textbf{Observaciones.}

    \begin{enumerate}
        \item $ K_n \ast K_m $ es bipartita completa $ \forall n, m \in \mathbb{N} $, pues $ K_n $ y $ K_m $ son $ (n-1) $-regular y $ (m-1) $-regular, respectivamente.
        \item $ C_n \ast C_m $ es bipartita completa $ \forall n, m \in \mathbb{N} $, pues $ C_n $ y $ C_m $ son gráficas $ 2 $-regular.
    \end{enumerate} \vspace{1mm}

    \begin{teorema}[beforeafter skip = 4mm]{}{Gg_igual_complementos}
        Si $ G_1 $ y $ G_2 $ son gráficas, entonces $ G_1 \ast G_2 = G_1^C \ast G_2^C $.
    \end{teorema}

    \begin{teorema}[beforeafter skip = 4mm]{}{trayectoria_minima}
        Sean $ G_1 $ y $ G_2 $ gráficas. Si para $ u,v \in \textnormal{V}(G_1) $ existe una $ uv $-trayectoria en $ G_1 \ast G_2 $ entonces existe una $ uv $-trayectoria $ T $ en $ G_1 \ast G_2 $ de longitud 2, con $ T = (u,w,v) $ donde $ w \in \textnormal{V}(G_2) $.
    \end{teorema}

    \begin{teorema}[beforeafter skip = 4mm]{}{trayectoria_minima}
        Sean $ G_1 $ y $ G_2 $ gráficas, con $ G_1 $ disconexa. Si $ \delta(G_1) = \delta(G_2) < \Delta(G_1) = \Delta(G_2) $ entonces $ G_1 \ast G_2 $ es disconexa.
    \end{teorema}

\end{document}